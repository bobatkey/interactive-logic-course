% -*- TeX-engine: xetex -*-

\documentclass[xetex,aspectratio=169,14pt,hyperref={pdfpagelabels=true,pdflang={en-GB}}]{beamer}

\usepackage[sci,noslidestrathidentity]{strathclyde}
\strathsetidentity{Department of}{Computer \& Information Sciences}

\usepackage{lmodern}
\usepackage{subscript}
\usepackage{url}
\usepackage{pifont}
\usepackage{csquotes}
\usepackage{mathpartir}
\usepackage{stmaryrd}
\usepackage{multirow}
\usepackage{euler}
\usepackage[normalem]{ulem}

% 2013-09-08 SU adding xetex specifics
% http://www.woggie.net/2008/07/16/beamer-pdftex-and-xetex/
\usepackage{xltxtra}

% 2013-09-08 SU http://robjhyndman.com/hyndsight/xelatex/
\defaultfontfeatures{Ligatures=TeX}

% 2015-12-09 SU table beautified
\renewcommand{\arraystretch}{1.2}
\usepackage{booktabs}

%MGK compatibility
\newcommand{\hh}[1]
  {\medskip\textbf{\large #1}}
\newcommand{\pdu}[3]
  {#1\rightarrow #2 : &\ & \makebox[70mm][l]{$#3$}}



\setlength{\marginparwidth}{2cm}
\usepackage{todonotes}%[disable]
\let\OldTodo\todo
\renewcommand{\todo}{\OldTodo[inline]}%
\newcommand{\todolater}[1]{}% Things to do for next year


\DeclareTextCommandDefault{\nobreakspace}{\leavevmode\nobreak\ }
\usepackage{rotating}

\usepackage{pdfcomment}% To add alt text for images using \pdftooltip{}

\usepackage{appendixnumberbeamer}

\newcommand{\messageframe}[1]{\begin{frame}\begin{center}\Huge #1\end{center}\end{frame}}
\newcommand{\sechead}[1]{{\bf #1} \\}
\newcommand{\examplehead}[1]{{\bf Example:} {\it\textcolor{red!90}{#1}} \\}
\newcommand{\eqnote}[1]{\hspace{3cm}\textit{#1}}
\newcommand{\sidenote}[1]{\qquad {\footnotesize \textcolor{black!60}{(#1)}}}
\newcommand{\sem}[1]{\llbracket #1 \rrbracket}
\newcommand{\true}{\mathsf{T}}
\newcommand{\false}{\mathsf{F}}

\def\strikeafter<#1>#2{\temporal<#1>{#2}{\sout{#2}}{\sout{#2}}}


%\newcommand{\rhighlight}{\textcolor{titlered}}
\newcommand{\rhighlight}{\textbf}
\newcommand{\highlight}{\textbf}

% \setmainfont{Linux Biolinum O}
\setmainfont{LinBiolinum}[
Path=,
UprightFont = *_R.otf ,
BoldFont = *_RB.otf ,
ItalicFont = *_RI.otf
]

\setbeamertemplate{navigation symbols}{}
%\usecolortheme[rgb={0.8,0,0}]{structure}
\usefonttheme{serif}
\usefonttheme{structurebold}
\setbeamercolor{description item}{fg=black}


\author[Atkey]{Dr.~Robert Atkey}
\institute[Strathclyde]{Computer \& Information Sciences}
\date[]{}

\newcommand{\weeksection}[1]{%
  \section{\thetitle{}, Part~\thesection : #1}
  \begin{frame}
    \begin{center}
      \textcolor{black!60}{\thetitle{}, Part \thesection}\\
      {\Huge #1}
    \end{center}
  \end{frame}}

\newcommand{\weektitle}[2]{\def\thetitle{#2}
\title[CS208 - Week #1]{CS208 (Semester 1) Week #1 : #2}}

\newcommand{\assigned}{:}

\newcommand{\forcedto}{\assigned_f}
\newcommand{\decideto}{\assigned_d}


\weektitle{3}{Logical Modelling II}

\begin{document}

\frame{\titlepage}

% \weeksection{Resource Allocation}

% \begin{frame}
%   {Resource Allocation}

%   Copy stuff from the exam paper.
% \end{frame}

\weeksection{Conversion to CNF}

\begin{frame}[t]
  {Conjunctive Normal Form (CNF)}

  \begin{displaymath}
    \begin{array}{cl}
      &(\lnot a \lor \lnot b \lor \lnot c) \\
      \land&(\lnot b \lor \lnot c \lor \lnot d) \\
      \land&(\lnot a \lor \lnot b \lor c) \\
      \land&b
    \end{array}
  \end{displaymath}
  \begin{enumerate}
  \item Entire formula is a conjunction $C_1 \land C_2 \land \cdots \land C_n$
  \item where each \emph{clause} $C_i = L_{i,1} \lor L_{i,2} \lor \cdots \lor L_{i,k}$
  \item where each \emph{literal} $L_{i,j} = x_{i,j}$ or $L_{i,j} = \lnot x_{i,j}$
  \end{enumerate}
\end{frame}

\begin{frame}[t]
  {Disjunctive Normal Form (DNF)}

  \emph{Disjunctive Normal Form} (DNF) is similar, but swaps $\land$
  and $\lor$.

  \begin{displaymath}
    \begin{array}{cl}
      &(\lnot a \land \lnot b \land \lnot c) \\
      \lor&(\lnot b \land \lnot c \land \lnot d) \\
      \lor&(\lnot a \land \lnot b \land c) \\
      \lor&b
    \end{array}
  \end{displaymath}
  \begin{enumerate}
  \item Entire formula is a \emph{dis}junction $D_1 \lor D_2 \lor \cdots \lor D_n$
  \item where each \emph{disjunct} $D_i = L_{i,1} \land L_{i,2} \land \cdots \land L_{i,k}$
  \item where each \emph{literal} $L_{i,j} = x_{i,j}$ or $L_{i,j} = \lnot x_{i,j}$
  \end{enumerate}
\end{frame}

\begin{frame}
  {Normal Forms and Satisfiability}

  \textbf{CNF} \\ Each clause is a \emph{constraint} and all constraints
  must be satisfied.

  \bigskip

  \textbf{DNF} \\ At least one of the disjuncts must be satisfied.

  \bigskip

  \emph{Exercise (after all the videos):} How would you write a SAT
  Solver for formulas in DNF? Why don't we do this instead of CNF?
\end{frame}

\begin{frame}
  {Conversion to CNF}

  Not every formula is in CNF, e.g.,
  \begin{displaymath}
    (A \land B) \to (B \land A)
  \end{displaymath}
  What if we want to use a SAT solver to determine satisfiability?

  \bigskip

  Two ways to convert a formula to CNF that is ``the same'':
  \begin{itemize}
  \item ``Multiplying out''
  \item Tseytin transformation
  \end{itemize}

  \bigskip

  First we need to define what we mean by ``the same''.

\end{frame}

\begin{frame}
  {Equivalent Formulas}
  Define two formulas $P$ and $Q$ to be \emph{equivalent}, written
  \begin{displaymath}
    P \equiv Q
  \end{displaymath}
  exactly when, for all valuations $v$,
  \begin{displaymath}
    \sem{P}v = \sem{Q}v
  \end{displaymath}
  \textcolor{black!60}{Equivalent to both $P \models Q$ and $Q \models P$ being valid}
\end{frame}

\begin{frame}
  {Simplifying Implication}

  \begin{displaymath}
    A \to B \equiv \lnot A \lor B
  \end{displaymath}
  \medskip
  \begin{displaymath}
    \begin{array}{cc|c|c|c}
      \multicolumn{2}{c|}{\textit{valuation}}&&P&Q \\
      A & B & \lnot A & A \to B & \lnot A \lor B \\
      \hline
      \false & \false & \true  & \true  & \true \\
      \false & \true  & \true  & \true  & \true \\
      \true  & \false & \false & \false & \false \\
      \true  & \true  & \false & \true  & \true
    \end{array}
  \end{displaymath}
\end{frame}

\begin{frame}[t]
  {Double Negation}

  Negating twice is the same as doing nothing:
  \begin{displaymath}
    A \equiv \lnot\lnot A
  \end{displaymath}
  \begin{displaymath}
    \begin{array}{c|c|c|c}
      \textit{valuation} & & P & Q \\
      A&\lnot A&A&\lnot \lnot A \\
      \hline
      \false&\true&\false&\false \\
      \true&\false&\true&\true
    \end{array}
  \end{displaymath}
\end{frame}

\begin{frame}
  {de Morgan's laws}

  Negation swaps $\land$ and $\lor$:
  \begin{displaymath}
    \lnot (A \land B) \equiv \lnot A \lor \lnot B
  \end{displaymath}
  \begin{displaymath}
    \begin{array}{cc|ccc|c|c}
      \multicolumn{2}{c|}{\textit{valuation}}&&&&P&Q\\
      A & B & \lnot A & \lnot B & A \land B & \lnot (A \land B) & \lnot A \lor \lnot B \\
      \hline
      \false & \false & \true  & \true  & \false & \true & \true \\
      \false & \true  & \true  & \false & \false & \true & \true \\
      \true  & \false & \false & \true  & \false & \true & \true \\
      \true  & \true  & \false & \false & \true  & \false & \false
    \end{array}
  \end{displaymath}
  Similar for $\lnot (A \lor B) \equiv \lnot A \land \lnot B$
\end{frame}

\begin{frame}
  {Negation Normal Form (NNF)}

  Using the equivalences:
  \begin{displaymath}
    \begin{array}{rcl}
      A \to B & \equiv & \lnot A \lor B \\
      A&\equiv&\lnot \lnot A \\
      \lnot (A \land B)&\equiv&\lnot A \lor \lnot B \\
      \lnot (A \lor B)&\equiv&\lnot A \land \lnot B
    \end{array}
  \end{displaymath}

  We can \emph{rewrite} any formula into an equivalent one with
  \begin{enumerate}
  \item No implications ($\to$s)
  \item All negation signs on the atomic propositions
  \end{enumerate}

  % \bigskip

  % Formula will have only $\land$s and $\lor$s, and $\lnot$s only at
  % the ``leaves''.
\end{frame}

\begin{frame}
  {Example}

  \begin{displaymath}
    \begin{array}{cll}
             & (a \land (a \to b)) \to c \\
      \equiv & \lnot (a \land (a \to b)) \lor c & \textit{converted }\to\\
      \equiv & \lnot (a \land (\lnot a \lor b)) \lor c & \textit{converted }\to\\
      \equiv & \lnot a \lor \lnot (\lnot a \lor b) \lor c & \textit{converted }\land\textit{ to }\lor \\
      \equiv & \lnot a \lor (\lnot \lnot a \land \lnot b) \lor c & \textit{converted }\lor\textit{ to }\land \\
      \equiv & \lnot a \lor (a \land \lnot b) \lor c & \textit{converted double negation} \\
    \end{array}
  \end{displaymath}

  \bigskip

  Now in NNF, but not CNF.
\end{frame}

\begin{frame}
  {``Push'' $\lor$s into $\land$s}
  \begin{displaymath}
    A \lor (B \land C) \equiv (A \lor B) \land (A \lor C)
  \end{displaymath}
  {\footnotesize\begin{displaymath}
    \begin{array}{ccc|ccc|c|c}
      \multicolumn{3}{c|}{\textit{valuation}}&&&&P&Q\\
      A & B & C       & B \land C & A \lor B & A \lor C & A \lor (B \land C) & (A \lor B) \land (A \lor C) \\
      \hline
      \false & \false & \false & \false       & \false      & \false      & \false                & \false \\
      \false & \false & \true  & \false       & \false      & \true       & \false                & \false \\
      \false & \true  & \false & \false       & \true       & \false      & \false                & \false \\
      \false & \true  & \true  & \true        & \true       & \true       & \true                 & \true  \\
      \true  & \false & \false & \false       & \true       & \true       & \true                 & \true  \\
      \true  & \false & \true  & \false       & \true       & \true       & \true                 & \true  \\
      \true  & \true  & \false & \false       & \true       & \true       & \true                 & \true  \\
      \true  & \true  & \true  & \true        & \true       & \true       & \true                 & \true  \\
    \end{array}
  \end{displaymath}}
\end{frame}

\begin{frame}
  {Conversion to CNF}

  \begin{displaymath}
    \begin{array}{cl}
             & \lnot a \lor (a \land \lnot b) \lor c \\
      \equiv & \hspace{4cm}\textit{multiply out} \\
             & \lnot a \lor ((a \lor c) \land (\lnot b \lor c)) \\
      \equiv & \hspace{4cm}\textit{multiply out} \\
             & (\lnot a \lor a \lor c) \land (\lnot a \lor \lnot b \lor c) \\
    \end{array}
  \end{displaymath}
  Now in CNF.

  \bigskip

  \textcolor{black!60}{(Can further simplify to: $(\lnot a \lor \lnot b \lor c)$)}
\end{frame}

\begin{frame}
  {Exponential Blowup}

  If we convert
  $(a \land b \land c) \lor (d \land e \land f) \lor (g \land h \land
  i)$ to CNF, we get:

  {\footnotesize
    \begin{displaymath}
      \begin{array}{@{}l}
        (a \lor d \lor g) \land (a \lor d \lor h) \land (a \lor d \lor i) \land (a \lor e \lor g) \land (a \lor e \lor h) \land \\
        (a \lor e \lor i) \land (a \lor f \lor g) \land (a \lor f \lor h) \land (a \lor f \lor i) \land (b \lor d \lor g) \land \\
        (b \lor d \lor h) \land (b \lor d \lor i) \land (b \lor e \lor g) \land (b \lor e \lor h) \land (b \lor e \lor i) \land \\
        (b \lor f \lor g) \land (b \lor f \lor h) \land (b \lor f \lor i) \land (c \lor d \lor g) \land (c \lor d \lor h) \land \\
        (c \lor d \lor i) \land (c \lor e \lor g) \land (c \lor e \lor h) \land (c \lor e \lor i) \land (c \lor f \lor g) \land \\
        (c \lor f \lor h) \land (c \lor f \lor i)
      \end{array}
    \end{displaymath}}

  which has $27$ clauses.

\end{frame}

\begin{frame}
  {Summary}

  \begin{itemize}
  \item SAT Solvers take their input in CNF
  \item Some problems are naturally in CNF
  \item Conversion by ``multiplying out'' can generate huge formulas
  \item We need something better
  \end{itemize}
\end{frame}

\weeksection{Tseytin Transformation}

\begin{frame}
  {Tseytin Transformation}

  The Tseytin transformation converts a formula into CNF with at most
  3 times as many clauses as connectives in the original formula
  (versus potentially exponential for multiplying out the brackets).

  \bigskip

  \begin{enumerate}
  \item Convert the formula into equations \\
    \quad \textcolor{black!60}{One connective $\leadsto$ one equation}
  \item Convert each equation into clauses \\
    \quad \textcolor{black!60}{One equation $\leadsto$ 2-3 clauses}
  \end{enumerate}

  \bigskip

  Result is not equivalent, but \emph{equisatisfiable}.


  % FIXME: give the basic idea

  % \textcolor{black!60}{Assume that all $\to$s have been removed}

  % If we can find values for $a, b, c$ \emph{and} $x_1, x_2, x_3$ such
  % that all these equations are true, and $x_1$ is true, then we've
  % found values for $a, b, c$ such that the original formula is
  % satisfied.

\end{frame}

\begin{frame}[t]
  {1. Name subformulas}

  Take the formula and name all the non-atomic subformulas.

  \medskip

  Example:
  \begin{displaymath}
    \lnot(a \land (\lnot a \lor b)) \lor c
  \end{displaymath}
  becomes:
  \begin{displaymath}
    \begin{array}{lcl}
      x_1&=&x_2 \lor c \\
      x_2&=&\lnot x_3 \\
      x_3&=&a \land x_4 \\
      x_4&=&x_5 \lor b\\
      x_5&=&\lnot a
    \end{array}
  \end{displaymath}
\end{frame}

\begin{frame}
  {2. Converting Equations to Clauses}

  Given an equation like $x = y \land z$, we want some clauses that
  are true for every valuation that satisfies the equation.

  \pause
  \bigskip

  Derive by conversion to CNF:
  \begin{displaymath}
    \begin{array}{cl}
      &x = y \land z \\
      \equiv&(x \to (y \land z)) \land ((y \land z) \to x) \\
      \equiv&(\lnot x \lor (y \land z)) \land (\lnot(y \land z) \lor x)\\
      \equiv&(\lnot x \lor y) \land (\lnot x \lor z) \land (\lnot y \lor \lnot z \lor x)
    \end{array}
  \end{displaymath}

\end{frame}

\begin{frame}
  {2. Equations to Clauses}

  Take each equation $x = y \mathop\Box z$ and turn it into clauses:
  \begin{enumerate}
  \item If $x = y \land z$, add
    \begin{displaymath}
      (\lnot x \lor y) \land (\lnot x \lor z) \land (\lnot y \lor \lnot z \lor x)
    \end{displaymath}
  \item If $x = y \lor z$, add
    \begin{displaymath}
      (y \lor z \lor \lnot x) \land (\lnot y \lor x) \land (\lnot z \lor x)
    \end{displaymath}
  \item If $x = \lnot y$, add
    \begin{displaymath}
      (\lnot y \lor \lnot x) \land (y \lor x)
    \end{displaymath}
  \end{enumerate}

\end{frame}

\begin{frame}
  {3. Assert the top level variable}

  If $x$ is the name of the whole formula, add a clause with just
  $x$:

  \begin{displaymath}
    \begin{array}{cl}
      &\textit{equation 1} \\
      \land&\textit{equation 2}\\
      \land&...\\
      \land&x
    \end{array}
  \end{displaymath}

  This asserts that our original formula must be true.
\end{frame}

\begin{frame}
  {Example: $\lnot (A \land B) \lor (B \land A)$}

  \rhighlight{1.}~Name the subformulas:
  \begin{displaymath}
    \begin{array}{lcl@{\hspace{2em}}lcl}
      x_1&=&x_2 \lor x_4 &
      x_2&=&\lnot x_3 \\
      x_3&=&A \land B &
      x_4&=&B \land A
    \end{array}
  \end{displaymath}

  \pause

  \rhighlight{2+3.}~Generate clauses: \textcolor{black!60}{(One line per equation)}
  \begin{displaymath}
    \begin{array}{cl}
      &(x_2 \lor x_4 \lor \lnot x_1) \land (\lnot x_2 \lor x_1) \land (\lnot x_4 \lor x_1) \\
      \land&(\lnot x_3 \lor \lnot x_2) \land (x_3 \lor x_2) \\
      \land&(\lnot A \lor \lnot B \lor x_3) \land (A \lor \lnot x_3) \land (B \lor \lnot x_3) \\
      \land&(\lnot B \lor \lnot A \lor x_4) \land (B \lor \lnot x_4) \land (A \lor \lnot x_4) \\
      \land&x_1
    \end{array}
  \end{displaymath}
\end{frame}

\begin{frame}
  {Efficiency}

  In small examples, we get many clauses.

  \bigskip

  But we \emph{always} get $\leq 3n$ clauses, where $n$ number of
  connectives.

  \bigskip

  Multiplying out can result in exponential number of clauses.

  \bigskip

  Can also optimise (see the tutorial questions).

\end{frame}

% \begin{frame}
%   A bigger example... FIXME
% \end{frame}

\begin{frame}
  {Not Equivalent!}

  The formulas generated by the Tseytin transformation are
  \textbf{not} equivalent to the original, because they have extra
  atomic propositions.
\end{frame}

\begin{frame}
  {Example}

  If the original formula is
  \begin{displaymath}
    \lnot A
  \end{displaymath}

  the Tseytin transformed version is: \textcolor{black!60}{(assuming
    we don't optimise)}
  \begin{displaymath}
    (\lnot A \lor \lnot x) \land (A \lor x) \land x
  \end{displaymath}

  Then $\{A \assigned \false, x \assigned \false\}$ satisfies the
  original, but not the transformed formula.
\end{frame}

\begin{frame}
  {Equisatisfiable}

  If we write $\mathsf{Tseytin}(P)$ for the Tseytin translation of
  $P$, then:
  \begin{enumerate}
  \item If there exists a valuation $v_1$ such that $\sem{P}v_1 = \true$,
    then there exists a valuation $v_2$ such that
    $\sem{\mathsf{Tseytin}(P)}v_2 = \true$;
  \item If there exists a valuation $v$ such that
    $\sem{\mathsf{Tseytin}(P)}v = \true$, then the valuation $v' = v$
    without the additional $x_i$s makes $\sem{P}v' = \true$.
  \end{enumerate}

  \bigskip

  This is called ``equisatisfiability''.
\end{frame}

\begin{frame}
  {Example}

  $v = \{A \assigned \false\}$ satisfies $\lnot A$

  \bigskip

  The corresponding satisfying valuation for
  \begin{displaymath}
    (\lnot A \lor \lnot x) \land (A \lor x) \land x
  \end{displaymath}
  is $\{A \assigned \false, x \assigned \true\}$.

  \bigskip

  A corresponding satisfying assignment always exists for the Tseytin
  transformation, because it is built from equations.
\end{frame}

\begin{frame}
  {Summary}

  \begin{itemize}
  \item Tseytin transformation converts formulas to CNF
  \item Generates $\leq 3n$ clauses, where $n$ is the number of connectives
  \item Avoids exponential blowup
  \item Can be further optimised
  \item Result is \emph{equisatisfiable}
  \end{itemize}
\end{frame}

\weeksection{Online Satisfiability Checker}

% \begin{frame}
%   {Checking Circuits}

% \end{frame}


\end{document}
