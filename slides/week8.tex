% -*- TeX-engine: xetex -*-

\documentclass[xetex,aspectratio=169,14pt,hyperref={pdfpagelabels=true,pdflang={en-GB}}]{beamer}

\usepackage[sci,noslidestrathidentity]{strathclyde}
\strathsetidentity{Department of}{Computer \& Information Sciences}

\usepackage{lmodern}
\usepackage{subscript}
\usepackage{url}
\usepackage{pifont}
\usepackage{csquotes}
\usepackage{mathpartir}
\usepackage{stmaryrd}
\usepackage{multirow}
\usepackage{euler}
\usepackage[normalem]{ulem}

% 2013-09-08 SU adding xetex specifics
% http://www.woggie.net/2008/07/16/beamer-pdftex-and-xetex/
\usepackage{xltxtra}

% 2013-09-08 SU http://robjhyndman.com/hyndsight/xelatex/
\defaultfontfeatures{Ligatures=TeX}

% 2015-12-09 SU table beautified
\renewcommand{\arraystretch}{1.2}
\usepackage{booktabs}

%MGK compatibility
\newcommand{\hh}[1]
  {\medskip\textbf{\large #1}}
\newcommand{\pdu}[3]
  {#1\rightarrow #2 : &\ & \makebox[70mm][l]{$#3$}}



\setlength{\marginparwidth}{2cm}
\usepackage{todonotes}%[disable]
\let\OldTodo\todo
\renewcommand{\todo}{\OldTodo[inline]}%
\newcommand{\todolater}[1]{}% Things to do for next year


\DeclareTextCommandDefault{\nobreakspace}{\leavevmode\nobreak\ }
\usepackage{rotating}

\usepackage{pdfcomment}% To add alt text for images using \pdftooltip{}

\usepackage{appendixnumberbeamer}

\newcommand{\messageframe}[1]{\begin{frame}\begin{center}\Huge #1\end{center}\end{frame}}
\newcommand{\sechead}[1]{{\bf #1} \\}
\newcommand{\examplehead}[1]{{\bf Example:} {\it\textcolor{red!90}{#1}} \\}
\newcommand{\eqnote}[1]{\hspace{3cm}\textit{#1}}
\newcommand{\sidenote}[1]{\qquad {\footnotesize \textcolor{black!60}{(#1)}}}
\newcommand{\sem}[1]{\llbracket #1 \rrbracket}
\newcommand{\true}{\mathsf{T}}
\newcommand{\false}{\mathsf{F}}

\def\strikeafter<#1>#2{\temporal<#1>{#2}{\sout{#2}}{\sout{#2}}}


%\newcommand{\rhighlight}{\textcolor{titlered}}
\newcommand{\rhighlight}{\textbf}
\newcommand{\highlight}{\textbf}

% \setmainfont{Linux Biolinum O}
\setmainfont{LinBiolinum}[
Path=,
UprightFont = *_R.otf ,
BoldFont = *_RB.otf ,
ItalicFont = *_RI.otf
]

\setbeamertemplate{navigation symbols}{}
%\usecolortheme[rgb={0.8,0,0}]{structure}
\usefonttheme{serif}
\usefonttheme{structurebold}
\setbeamercolor{description item}{fg=black}


\author[Atkey]{Dr.~Robert Atkey}
\institute[Strathclyde]{Computer \& Information Sciences}
\date[]{}

\newcommand{\weeksection}[1]{%
  \section{\thetitle{}, Part~\thesection : #1}
  \begin{frame}
    \begin{center}
      \textcolor{black!60}{\thetitle{}, Part \thesection}\\
      {\Huge #1}
    \end{center}
  \end{frame}}

\newcommand{\weektitle}[2]{\def\thetitle{#2}
\title[CS208 - Week #1]{CS208 (Semester 1) Week #1 : #2}}

\newcommand{\assigned}{:}

\newcommand{\forcedto}{\assigned_f}
\newcommand{\decideto}{\assigned_d}


\weektitle{8}{Predicate Logic: Semantics}

\begin{document}

\frame{\titlepage}

\weeksection{Models}

\begin{frame}
  {So far: Syntax and Proof}

  \begin{enumerate}
  \item The syntax of predicate logic \\
    \textcolor{black!60}{What sequences of symbols are well formed?}
  \item Proofs for predicate logic \\
    \textcolor{black!60}{When are formulas consequences of other formulas?}
  \end{enumerate}
\end{frame}

\begin{frame}
  {Missing so far: \emph{semantics}}
  \begin{enumerate}
  \item For Propositional Logic, we defined the \emph{semantics}
    (``meaning'') of a formula $P$:
    \begin{itemize}
    \item For every \emph{valuation} $v$,\\
      the formula $P$ is assigned a meaning $\sem{P}~v$ which is either $\true$ or $\false$.
    \end{itemize}
  \item This definition enabled us to give a definition of
    \emph{entailment}:
    \begin{displaymath}
      P_1, \dots, P_n \models Q
    \end{displaymath}
    which defines consequence without using proofs.
  \end{enumerate}
\end{frame}

\begin{frame}
  {Semantics for Predicate Logic}

  The plan:
  \begin{enumerate}
  \item Fix a vocabulary
  \item Define \emph{models} $\mathcal{M}$
  \item Interpret a formula $P$ in a model $\mathcal{M}$
  \end{enumerate}
\end{frame}

\begin{frame}
  {Fixing a Vocabulary}

  The function symbols we will use, and their \emph{arities}
  (number of arguments):
  \begin{displaymath}
    \begin{array}{l|l}
      \textrm{Function name(s)}&\textrm{Arity}\\
      \hline
      \mathrm{socrates} & 0 \\
      \mathrm{dayAfter} & 1 \\
      +, -              & 2\\
    \end{array}
  \end{displaymath}
  We write ``$\mathrm{func}/n$'' for function symbol $\mathrm{func}$
  with arity $n$
\end{frame}

\begin{frame}
  {Fixing a Vocabulary}

  The predicates / relation symbols we will use, and their arities:
  \begin{displaymath}
    \begin{array}{l|l}
      \textrm{Predicate name(s)}&\textrm{Arity} \\
      \hline
      \mathrm{human}, \mathrm{mortal} & 1 \\
      <, \leq, =                      & 2 \\
      \mathrm{between}                & 3
    \end{array}
  \end{displaymath}
  We write ``$\mathrm{pred}/n$'' for predicate symbol
  $\mathrm{pred}$ with arity $n$
\end{frame}

\begin{frame}
  \sechead{A simplification}

  \bigskip

  To keep things simple, I'm going to assume that we don't have any
  function symbols in our vocabulary.
\end{frame}

\begin{frame}
  \sechead{Example: Orderings}
  \begin{itemize}
  \item $\mathord{\leq}/2$ \quad \textcolor{black!60}{``less than''}
  \end{itemize}

  \bigskip

  \sechead{Example: Places}
  \begin{itemize}
  \item $\mathrm{city}/1$ \quad \textcolor{black!60}{``is a city''}
  \item $\mathrm{within}/2$ \quad \textcolor{black!60}{``is within''}
  \end{itemize}

  \bigskip

  \sechead{Example: Forestry and Birdwatching}
  \begin{itemize}
  \item $\mathrm{tree}/1$ \quad \textcolor{black!60}{``is a tree''}
  \item $\mathrm{green}/1$ \quad \textcolor{black!60}{``is green''}
  \item $\mathrm{bird}/1$ \quad \textcolor{black!60}{``is a bird''}
  \item $\mathrm{satIn}/2$ \quad \textcolor{black!60}{``has sat in''}
  \end{itemize}
\end{frame}

\begin{frame}
  {Models}

  With a fixed vocabulary, a \emph{model} $\mathcal{M}$ is:
  \begin{enumerate}
  \item A \emph{universe} $U$, which is a set of individuals:
    \begin{displaymath}
      U = \{ 1, 2, \mathsf{socrates}, \mathsf{hypatia}, \mathsf{noether}, \mathsf{alexandria}, \mathsf{glasgow}, \dots \}
    \end{displaymath}
  \item For each predicate $\mathrm{pred}/n$, an $n$-ary
    relation on the set $U$.
  \end{enumerate}
\end{frame}

\begin{frame}
  {Relations}

  Several ways of understanding what a relation is:
  \begin{enumerate}
  \item For every $n$ elements from $U$, the interpretation of
    $\mathrm{pred}/n$ assigns the value $\true$ or $\false$.\\
    \bigskip
  \item The interpretation of $\mathrm{pred}/n$ is a (possibly
    infinite) table of elements of $U$ with $n$ columns.
    \bigskip
  \item The interpretation of $\mathrm{pred}/n$ is a subset of the
    $n$-fold \emph{cartesian product}
    $\underbrace{U \times \dots \times U}_{n~\mathrm{times}}$.\\
  \end{enumerate}
\end{frame}

\begin{frame}
  \sechead{Example: Places, interpretation 1}
  \begin{displaymath}
    \begin{array}{l@{\hspace{0.3em}}c@{\hspace{0.3em}}l}
      U &=& \{ \mathsf{aberdeen}, \mathsf{edinburgh}, \mathsf{glasgow}, \mathsf{scotland},  \mathsf{birmingham}, \mathsf{england} \}\\
      \mathsf{city} &=& \{ (\mathsf{aberdeen}), (\mathsf{edinburgh}), (\mathsf{glasgow}), (\mathsf{birmingham}) \}\\
      \mathsf{within} &=& \{
                          \begin{array}[t]{@{}l}
                            (\mathsf{aberdeen}, \mathsf{scotland}), (\mathsf{edinburgh}, \mathsf{scotland}), \\
                            (\mathsf{glasgow}, \mathsf{scotland}), (\mathsf{birmingham}, \mathsf{england}) \}
                          \end{array}
    \end{array}
  \end{displaymath}
\end{frame}

\begin{frame}
  \sechead{Example: Places, interpretation 1}

  \bigskip

  \begin{displaymath}
    U = \{ \mathsf{aberdeen}, \mathsf{edinburgh}, \mathsf{glasgow}, \mathsf{scotland},  \mathsf{birmingham}, \mathsf{england} \}
  \end{displaymath}
  As tables:
  \begin{mathpar}
    \begin{array}{|c|}
      \hline
      \mathrm{city} \\
      \hline
      \mathsf{aberdeen} \\
      \mathsf{edinburgh} \\
      \mathsf{glasgow} \\
      \mathsf{birmingham}\\
      \hline
    \end{array}

    \begin{array}{|cc|}
      \hline
      \multicolumn{2}{|c|}{\mathrm{within}}\\
      \hline
      \mathsf{aberdeen} & \mathsf{scotland} \\
      \mathsf{edinburgh} & \mathsf{scotland} \\
      \mathsf{glasgow} & \mathsf{scotland} \\
      \mathsf{birmingham} & \mathsf{england} \\
      \hline
    \end{array}
  \end{mathpar}
\end{frame}

\begin{frame}
  \sechead{Example: Places, interpretation 2}
  \begin{displaymath}
    \begin{array}{lcl}
      U &=& \{ \textsf{planet-b} \} \\
      \mathsf{city} &=& \{ \} \\
      \mathsf{within} &=& \{ (\textsf{planet-b}, \textsf{planet-b}) \}
    \end{array}
  \end{displaymath}
\end{frame}

\begin{frame}
  \sechead{Example: Interpreting ordering with natural numbers}
  \begin{enumerate}
  \item $U = \{ 0, 1, 2, \dots \} = \mathbb{N}$ (all positive whole numbers)
  \item The interpretation of $\mathord{\leq}/2$ is all pairs $(x, y)$ such that $x \leq y$
  \end{enumerate}
\end{frame}

\begin{frame}
  \sechead{Example: Interpreting ordering with rational numbers}
  \begin{enumerate}
  \item $U = \{ 0, -1, 1, -\frac{1}{2}, \frac{1}{2}, -2, 2, \dots \} = \mathbb{Q}$
  \item The interpretation of $\mathord{\leq}/2$ is all pairs $(x, y)$ such that $x \leq y$
  \end{enumerate}
\end{frame}

\begin{frame}
  \sechead{Example: Interpreting ordering with a small set}
  \begin{enumerate}
  \item $U = \{ \mathsf{a}, \mathsf{b}, \mathsf{c} \}$
  \item The interpretation of $\mathord{\leq}/2$ is the set:
    \begin{displaymath}
      \{ (\mathsf{a},\mathsf{b}), (\mathsf{a},\mathsf{c}) \}
    \end{displaymath}
    Note! not necessarily what we might think of as $\leq$! Need to
    add axioms.
  \end{enumerate}
\end{frame}

\begin{frame}
  {Important Points}


  Every model $\mathcal{M}$ has
  \begin{enumerate}
  \item a universe; and
  \item a relation for each predicate symbol $\mathrm{pred}/n$,
  \end{enumerate}
  but the domain can be empty, or the predicate symbols' interpretations may be empty!

  \bigskip

  The model needn't match our intuition about the symbols!
  \begin{itemize}
  \item Will assume formulas that will filter the possible models.
  \end{itemize}
\end{frame}

\begin{frame}
  {Relationship to Valuations}

  \bigskip

  If all our predicate symbols have arity $0$ (take no arguments),
  then a model consists of:
  \begin{enumerate}
  \item A universe $U$; and
  \item An assignment of $\true$ or $\false$ to each predicate symbol
    $\mathrm{pred}/0$.
  \end{enumerate}

  \bigskip

  Apart from the universe, this is the same as a \emph{valuation} in
  Propositional Logic (Week 01).
\end{frame}

\begin{frame}
  {Summary}

  We interpret Predicate Logic formulas in a \emph{model} $\mathcal{M}$.
  \begin{itemize}
  \item A universe $U$ -- the set of all ``things''.
  \item A relation between elements of $U$ for every predicate.
  \end{itemize}

  \bigskip

  Useful intuition: models are (possibly infinite) databases.
\end{frame}

%%%%%%%%%%%%%%%%%%%%%%%%%%%%%%%%%%%%%%%%%%%%%%%%%%%%%%%%%%%%%%%%%%%%%%%%%%%%%%
%%%%%%%%%%%%%%%%%%%%%%%%%%%%%%%%%%%%%%%%%%%%%%%%%%%%%%%%%%%%%%%%%%%%%%%%%%%%%%
%%%%%%%%%%%%%%%%%%%%%%%%%%%%%%%%%%%%%%%%%%%%%%%%%%%%%%%%%%%%%%%%%%%%%%%%%%%%%%
\weeksection{Interpreting Formulas}

\begin{frame}
  {Meaning of free variables}

  Assume a vocabulary $\mathcal{V}$ and model $\mathcal{M}$ are fixed.

  \bigskip

  Consider the formula:
  \begin{displaymath}
    \mathrm{city}(x) \land \mathrm{within}(x,y)
  \end{displaymath}
  we can't give it a truth value until we know what $x$ and $y$ mean.
\end{frame}

\begin{frame}
  {Cities Model}
  \begin{displaymath}
    U = \{ \mathsf{aberdeen}, \mathsf{edinburgh}, \mathsf{glasgow}, \mathsf{scotland},  \mathsf{birmingham}, \mathsf{england} \}
  \end{displaymath}
  As tables:
  \begin{mathpar}
    \begin{array}{|c|}
      \hline
      \mathrm{city} \\
      \hline
      \mathsf{aberdeen} \\
      \mathsf{edinburgh} \\
      \mathsf{glasgow} \\
      \mathsf{birmingham}\\
      \hline
    \end{array}

    \begin{array}{|cc|}
      \hline
      \multicolumn{2}{|c|}{\mathrm{within}}\\
      \hline
      \mathsf{aberdeen} & \mathsf{scotland} \\
      \mathsf{edinburgh} & \mathsf{scotland} \\
      \mathsf{glasgow} & \mathsf{scotland} \\
      \mathsf{birmingham} & \mathsf{england} \\
      \hline
    \end{array}
  \end{mathpar}
\end{frame}

\begin{frame}
  {Meaning of free variables}

  With the cities model, if we set:
  \begin{mathpar}
    x = \mathsf{glasgow}

    y = \mathsf{scotland}
  \end{mathpar}
  then $\mathrm{city}(x) \land \mathrm{within}(x,y)$ should be assigned the truth value $\true$.

  \bigskip
  \pause

  If we set:
  \begin{mathpar}
    x = \mathsf{scotland}

    y = \mathsf{edinburgh}
  \end{mathpar}
  then $\mathrm{city}(x) \land \mathrm{within}(x,y)$ should be assigned the truth value $\false$.
\end{frame}

\begin{frame}
  {Interpreting Formulas}

  If we fix:
  \begin{enumerate}
  \item a vocabulary $\mathcal{V}$;
  \item a model $\mathcal{M}$ of that vocabulary;
  \item an assignment $v$ of elements of $U$ to free variables of $P$.
  \end{enumerate}
  then we can give a truth value $\sem{P}({\mathcal{M}},v)$ to $P$.
\end{frame}

\begin{frame}
  {Interpreting Formulas}

  Relations:
  \begin{displaymath}
    \begin{array}{lcl}
      \sem{R(x_1, \dots, x_n)}(\mathcal{M},v) &=& \true \quad \mathrm{if}\quad (v(x_1), \dots, v(x_n)) \in \mathrm{R} \\
                                              &=& \false \quad \textrm{otherwise} \\
      \sem{x = y}(\mathcal{M},v)              &=& \true \quad \textrm{if}\quad v(x) = v(y) \\
                                              &=& \false \quad \textrm{otherwise}
    \end{array}
  \end{displaymath}
  where $R$ is one of the relations in $\mathcal{M}$.
\end{frame}

\begin{frame}
  {Interpreting Formulas (Example)}

  With the cities model $\mathcal{M}$:

  \bigskip

  \begin{displaymath}
    \sem{\mathrm{within}(x,y)}(\mathcal{M}, [x\mapsto \mathsf{edinburgh}, y \mapsto \mathsf{scotland}]) = \true
  \end{displaymath}

  \bigskip

  \begin{displaymath}
    \sem{\mathrm{within}(x,y)}(\mathcal{M}, [x\mapsto \mathsf{edinburgh}, y \mapsto \mathsf{england}]) = \false
  \end{displaymath}
\end{frame}

\begin{frame}
  {Interpreting Formulas}

  Quantifiers:
  \begin{displaymath}
    \begin{array}{lcl}
      \sem{\forall x. P}(\mathcal{M},v) & = & \true \quad\textrm{if for all }a \in U,~\sem{P}(\mathcal{M},v[x \mapsto a]) = \true \\
      & = & \false \quad \textrm{otherwise} \\
      \sem{\exists x. P}(\mathcal{M},v) & = & \true \quad\textrm{if exists }a \in U,\textrm{ with }\sem{P}(\mathcal{M},v[x \mapsto a]) = \true \\
      & = & \false \quad \textrm{otherwise} \\
    \end{array}
  \end{displaymath}
  Notation $v[x \mapsto a]$ means the assignment that maps $x$ to $a$
  and any other variable to whatever $v$ mapped it to.
\end{frame}

\begin{frame}
  {Interpreting Formulas (Example)}

  \begin{displaymath}
    \sem{\forall x. \mathrm{city}(x)}(\mathcal{M},[]) = \false
  \end{displaymath}
  because all of the following would need to be $\true$:
  \begin{displaymath}
    \begin{array}{lcc}
      \sem{\mathrm{city}(x)}(\mathcal{M},[x\mapsto\mathsf{aberdeen}]) &=& \true\\
      \sem{\mathrm{city}(x)}(\mathcal{M},[x\mapsto\mathsf{edinburgh}]) &=& \true\\
      \sem{\mathrm{city}(x)}(\mathcal{M},[x\mapsto\mathsf{glasgow}]) &=& \true\\
      \sem{\mathrm{city}(x)}(\mathcal{M},[x\mapsto\mathsf{birmingham}]) &=& \true\\
      \sem{\mathrm{city}(x)}(\mathcal{M},[x\mapsto\mathsf{scotland}]) &=& \false\\
      \sem{\mathrm{city}(x)}(\mathcal{M},[x\mapsto\mathsf{england}]) &=& \false\\
    \end{array}
  \end{displaymath}
\end{frame}

\begin{frame}
  {Interpreting Formulas (Example)}

  \begin{displaymath}
    \sem{\exists x. \mathrm{city}(x)}(\mathcal{M},[]) = \true
  \end{displaymath}
  because only one of the following needs to be $\true$:
  \begin{displaymath}
    \begin{array}{lcc}
      \sem{\mathrm{city}(x)}(\mathcal{M},[x\mapsto\mathsf{aberdeen}]) &=& \true\\
      \sem{\mathrm{city}(x)}(\mathcal{M},[x\mapsto\mathsf{edinburgh}]) &=& \true\\
      \sem{\mathrm{city}(x)}(\mathcal{M},[x\mapsto\mathsf{glasgow}]) &=& \true\\
      \sem{\mathrm{city}(x)}(\mathcal{M},[x\mapsto\mathsf{birmingham}]) &=& \true\\
      \sem{\mathrm{city}(x)}(\mathcal{M},[x\mapsto\mathsf{scotland}]) &=& \false\\
      \sem{\mathrm{city}(x)}(\mathcal{M},[x\mapsto\mathsf{england}]) &=& \false\\
    \end{array}
  \end{displaymath}
\end{frame}

\begin{frame}
  {Interpreting Formulas}

  Propositional Connectives:
  \begin{displaymath}
    \begin{array}{lcl}
      \sem{P \land Q}(\mathcal{M},v) &=& \sem{P}(\mathcal{M},v) \land \sem{Q}(\mathcal{M},v)\\
      \sem{P \lor Q}(\mathcal{M},v) &=& \sem{P}(\mathcal{M},v) \lor \sem{Q}(\mathcal{M},v)\\
      \sem{P \to Q}(\mathcal{M},v) &=& \sem{P}(\mathcal{M},v) \to \sem{Q}(\mathcal{M},v)\\
      \sem{\lnot P}(\mathcal{M},v) &=& \lnot \sem{P}(\mathcal{M},v)  \\
    \end{array}
  \end{displaymath}
\end{frame}

\begin{frame}
  {Interpreting Formulas (Example)}

  \begin{displaymath}
    \sem{\mathrm{city}(x) \land \mathrm{within}(x,y)}(\mathcal{M},[x\mapsto\mathsf{edinburgh},y\mapsto\mathsf{scotland}]) = \true
  \end{displaymath}
  and
  \begin{displaymath}
    \sem{\mathrm{city}(x) \land \mathrm{within}(x,y)}(\mathcal{M},[x\mapsto\mathsf{edinburgh},y\mapsto\mathsf{birmingham}]) = \false
  \end{displaymath}
  and
  \begin{displaymath}
    \sem{\mathrm{city}(x) \lor \mathrm{within}(x,y)}(\mathcal{M},[x\mapsto\mathsf{edinburgh},y\mapsto\mathsf{birmingham}]) = \true
  \end{displaymath}
\end{frame}

\begin{frame}
  {Some notation}

  We write
  \begin{displaymath}
    \mathcal{M} \models P
  \end{displaymath}
  when
  \begin{displaymath}
    \sem{P}(\mathcal{M}, []) = \true
  \end{displaymath}
  meaning that $\mathcal{M}$ is a model of $P$.
\end{frame}

\begin{frame}
  {Examples}

  If $\mathcal{M}$ is the cities model, then
  \begin{displaymath}
    \mathcal{M} \models \exists x. \mathrm{city}(x)
  \end{displaymath}
  and
  \begin{displaymath}
    \mathcal{M} \not\models \forall x. \mathrm{city}(x)
  \end{displaymath}
  and
  \begin{displaymath}
    \mathcal{M} \models \forall x. \mathrm{city}(x) \to (\exists y. \mathrm{within}(x,y))
  \end{displaymath}
\end{frame}

\begin{frame}
  {Entailment}

  Relative to a model $\mathcal{M}$:
  \begin{displaymath}
    \mathcal{M}; P_1, \dots, P_n \models Q
  \end{displaymath}
  exactly when:
  \begin{center}
    if all $\sem{P_i}(\mathcal{M},[]) = \true$, then $\sem{Q}(\mathcal{M},[]) = \true$.
  \end{center}

  \bigskip

  \textcolor{black!60}{If all the assumptions are true, then the conclusion must be true}
\end{frame}

\begin{frame}
  {Entailment}

  \begin{displaymath}
    P_1, \dots, P_n \models Q
  \end{displaymath}
  exactly when \emph{for all} $\mathcal{M}$, we have $\mathcal{M};P_1,\dots,P_n \models Q$.

  \bigskip

  Checking this is infeasible (at least naively): there are infinitely
  many models, and the models themselves may be infinite.

  \bigskip

  Which is one reason to use proof.
\end{frame}

\begin{frame}
  {Axiomatisations}

  Some collections of formulas are called \emph{axiomatisations}.

  \bigskip

  For example,
  \begin{displaymath}
    \begin{array}{l}
      \forall x. x \leq x \\
      \forall x. \forall y. (x \leq y \land y \leq x) \to x = y \\
      \forall x. \forall y. \forall z. (x \leq y \land y \leq z) \to x \leq z
    \end{array}
  \end{displaymath}
  axiomatises what it means to be a \emph{partial order}.
\end{frame}

\begin{frame}
  {Axiomatisations}

  If we write $\Gamma_{\mathit{partial}}$ for those formulas $\land$ed
  together, then collection of models $\mathcal{M}$ such that:
  \begin{displaymath}
    \mathcal{M} \models \Gamma_{\mathit{partial}}
  \end{displaymath}
  is the collection of partial orders.

  \bigskip

  And the collection of formulas $Q$ such that
  \begin{displaymath}
    \Gamma_{\mathit{partial}} \models Q
  \end{displaymath}
  is the collection of facts that are true about all partial orders.
\end{frame}

\begin{frame}
  {Summary}

  We have defined what it means for a Predicate Logic $P$ formula to
  be true in some model $\mathcal{M}$.

  \begin{itemize}
  \item Just as with Propositional Logic, this is done by breaking
    down the formula into its constituent parts
  \item Care must be taken to ensure that all free variables have an
    interpretation.
  \item When a formula is true in some model, we write:
    \begin{displaymath}
      \mathcal{M} \models P
    \end{displaymath}
  \item Gives us a basis to talk about axiomatisations.
  \end{itemize}
\end{frame}

\weeksection{Model Checking Formulas}


% \weeksection{Modelling Software}

% \begin{frame}
%   FIXME: an application of predicate logic
% \end{frame}

\end{document}
