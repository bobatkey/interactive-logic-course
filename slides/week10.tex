% -*- TeX-engine: xetex -*-

\documentclass[xetex,aspectratio=169,14pt,hyperref={pdfpagelabels=true,pdflang={en-GB}}]{beamer}

\usepackage[sci,noslidestrathidentity]{strathclyde}
\strathsetidentity{Department of}{Computer \& Information Sciences}

\usepackage{lmodern}
\usepackage{subscript}
\usepackage{url}
\usepackage{pifont}
\usepackage{csquotes}
\usepackage{mathpartir}
\usepackage{stmaryrd}
\usepackage{multirow}
\usepackage{euler}
\usepackage[normalem]{ulem}

% 2013-09-08 SU adding xetex specifics
% http://www.woggie.net/2008/07/16/beamer-pdftex-and-xetex/
\usepackage{xltxtra}

% 2013-09-08 SU http://robjhyndman.com/hyndsight/xelatex/
\defaultfontfeatures{Ligatures=TeX}

% 2015-12-09 SU table beautified
\renewcommand{\arraystretch}{1.2}
\usepackage{booktabs}

%MGK compatibility
\newcommand{\hh}[1]
  {\medskip\textbf{\large #1}}
\newcommand{\pdu}[3]
  {#1\rightarrow #2 : &\ & \makebox[70mm][l]{$#3$}}



\setlength{\marginparwidth}{2cm}
\usepackage{todonotes}%[disable]
\let\OldTodo\todo
\renewcommand{\todo}{\OldTodo[inline]}%
\newcommand{\todolater}[1]{}% Things to do for next year


\DeclareTextCommandDefault{\nobreakspace}{\leavevmode\nobreak\ }
\usepackage{rotating}

\usepackage{pdfcomment}% To add alt text for images using \pdftooltip{}

\usepackage{appendixnumberbeamer}

\newcommand{\messageframe}[1]{\begin{frame}\begin{center}\Huge #1\end{center}\end{frame}}
\newcommand{\sechead}[1]{{\bf #1} \\}
\newcommand{\examplehead}[1]{{\bf Example:} {\it\textcolor{red!90}{#1}} \\}
\newcommand{\eqnote}[1]{\hspace{3cm}\textit{#1}}
\newcommand{\sidenote}[1]{\qquad {\footnotesize \textcolor{black!60}{(#1)}}}
\newcommand{\sem}[1]{\llbracket #1 \rrbracket}
\newcommand{\true}{\mathsf{T}}
\newcommand{\false}{\mathsf{F}}

\def\strikeafter<#1>#2{\temporal<#1>{#2}{\sout{#2}}{\sout{#2}}}


%\newcommand{\rhighlight}{\textcolor{titlered}}
\newcommand{\rhighlight}{\textbf}
\newcommand{\highlight}{\textbf}

% \setmainfont{Linux Biolinum O}
\setmainfont{LinBiolinum}[
Path=,
UprightFont = *_R.otf ,
BoldFont = *_RB.otf ,
ItalicFont = *_RI.otf
]

\setbeamertemplate{navigation symbols}{}
%\usecolortheme[rgb={0.8,0,0}]{structure}
\usefonttheme{serif}
\usefonttheme{structurebold}
\setbeamercolor{description item}{fg=black}


\author[Atkey]{Dr.~Robert Atkey}
\institute[Strathclyde]{Computer \& Information Sciences}
\date[]{}

\newcommand{\weeksection}[1]{%
  \section{\thetitle{}, Part~\thesection : #1}
  \begin{frame}
    \begin{center}
      \textcolor{black!60}{\thetitle{}, Part \thesection}\\
      {\Huge #1}
    \end{center}
  \end{frame}}

\newcommand{\weektitle}[2]{\def\thetitle{#2}
\title[CS208 - Week #1]{CS208 (Semester 1) Week #1 : #2}}

\newcommand{\assigned}{:}

\newcommand{\forcedto}{\assigned_f}
\newcommand{\decideto}{\assigned_d}


\weektitle{10}{Metatheory of Predicate Logic}

\begin{document}

\frame{\titlepage}

\weeksection{Metatheory}

\begin{frame}
  \begin{enumerate}
  \item Is our proof system sound and complete?
  \item Are our axiomatisations complete enough?
  \item Can we automate mathematics?
  \end{enumerate}
\end{frame}

\begin{frame}
  {Soundness}

  The proof system we have seen so far is \emph{sound}:
  \begin{displaymath}
    \Gamma \vdash Q \quad \Rightarrow \quad \Gamma \models Q
  \end{displaymath}
  \textcolor{black!60}{``Every provable judgement is valid.''}\\
  \textcolor{black!60}{Can be checked by checking that every rule preserves validity.}
\end{frame}

\begin{frame}
  {Completeness}

  If we add a rule for excluded middle ($P \lor \lnot P$ for any formula $P$), then it is \emph{complete}:
  \begin{displaymath}
    \Gamma \models Q \quad \Rightarrow \quad \Gamma \vdash Q
  \end{displaymath}
  \textcolor{black!60}{``Every valid judgement is provable.''}\\
  \textcolor{black!60}{This is not a simple fact. ``G{\"o}del's Completeness Theorem''}
\end{frame}

\begin{frame}
  {Automating Mathematics?}

  If our proof system is sound and complete, then we should be able to
  automatically prove things by searching for proofs?

  \bigskip

  This is one of the oldest branches of AI.
\end{frame}

\begin{frame}
  {Automating Mathematics?}

  We have seen so far that there are many axiomatisations for
  describing certain bits of mathematics:
  \begin{enumerate}
  \item Monoid axioms: addition with a zero.
  \item Peano's axioms: arithmetic
  \item Zermelo-Frankel axioms: set theory
  \end{enumerate}

  \bigskip

  Are these axiomatisations suitable for finding proofs?
\end{frame}

\begin{frame}
  {Syntactic Completeness}

  An axiomatisation $\mathrm{Ax}$ is \emph{syntactically complete} if
  for all formulas $P$, we can prove one of:
  \begin{displaymath}
    \mathrm{Ax} \vdash P
  \end{displaymath}
  or
  \begin{displaymath}
    \mathrm{Ax} \vdash \lnot P
  \end{displaymath}
  if we can prove both, then the theory is \emph{inconsistent}.
\end{frame}

\begin{frame}
  {Effectively Generatable}

  An axiomatisation $\mathrm{Ax}$ is \emph{effectively generatable} if
  we can write a computer program that generates all the valid axioms.

  \bigskip

  There may be infinitely many axioms, but each one will eventually be
  generated.
\end{frame}

\begin{frame}
  {Automation}

  If an axiomatisation $\mathrm{Ax}$ is syntactically complete and
  effectively generatable, then we can (in principle) write a program
  to search for a proof of some $P$:

  \bigskip

  \begin{enumerate}
  \item Search for a proof $\mathrm{Ax} \vdash P$ \\
    try proofs of size 1, then proofs of size 2, then proofs of size 3...
  \item Interleaved with this: search for a proof of $\mathrm{Ax} \vdash \lnot P$
  \end{enumerate}

  \bigskip

  Since one of them is provable, we will eventually terminate.
\end{frame}

\begin{frame}
  \begin{center}
    Is every interesting axiomatisation syntactically complete?
  \end{center}
\end{frame}

\begin{frame}
  {Peano's axioms ($\mathrm{PA}$)}

  \begin{enumerate}
  \item $\forall x.~\lnot(0 = S(x))$
  \item $\forall x. \forall y.~S(x) = S(y) \to x = y$
  \item $\forall x.~add(0,x) = x$
  \item $\forall x.\forall y.~add(S(x),y) = S(add(x,y))$
  \item $\forall x.~mul(0,x) = 0$
  \item $\forall x. \forall y.~mul(S(x),y) = add(y,mul(x,y))$
  \end{enumerate}
  $+$ induction

  \bigskip

  is effectively generatable.
\end{frame}

\begin{frame}
  {Robinson's axioms ($\mathrm{Q}$)}

  \begin{enumerate}
  \item $\forall x.~\lnot(0 = S(x))$
  \item $\forall x. \forall y.~S(x) = S(y) \to x = y$
  \item $\forall x.~add(0,x) = x$
  \item $\forall x.\forall y.~add(S(x),y) = S(add(x,y))$
  \item $\forall x.~mul(0,x) = 0$
  \item $\forall x. \forall y.~mul(S(x),y) = add(y,mul(x,y))$
  \item $\forall x. (x = 0) \lor (\exists y. x = S(y))$\qquad {\bf (instead of induction)}
  \end{enumerate}
\end{frame}

\begin{frame}
  {G{\"o}del's 1st Incompleteness Theorem}

  For any \emph{effectively generatable} consistent set of axioms
  $\mathrm{Ax}$ that imply those of Robinson arithmetic, there exists
  a formula $P$ such that it is \textbf{not} possible to prove either
  of
  \begin{displaymath}
    \mathrm{Ax} \vdash P
  \end{displaymath}
  or
  \begin{displaymath}
    \mathrm{Ax} \vdash \lnot P
  \end{displaymath}
\end{frame}

\begin{frame}
  {Consequences}

  $\mathrm{PA}$ is not syntactically complete, so our attempt to use
  it to automate mathematics fails.

  \bigskip

  In fact, provability in $\mathrm{PA}$ is undecidable, so all attempts
  are doomed.
\end{frame}

\begin{frame}
  {Consequences}

  \begin{enumerate}
  \item $\mathrm{PA}$ is incomplete, so there is a formula $P$ such that neither of:
    \begin{mathpar}
      \mathrm{PA} \vdash P

      \textrm{and}

      \mathrm{PA} \vdash \lnot P
    \end{mathpar}
    are provable.
  \item Inspection of G{\"o}del's proof shows that the formula $P$ it
    generates is actually true in ``the'' natural numbers.
  \item So we could use the axioms $\mathrm{PA} + P$, but then goto 1.
  \end{enumerate}
\end{frame}

% https://stopa.io/post/269

\begin{frame}
  {Consequences}

  So:
  \begin{enumerate}
  \item $\mathrm{PA}$ does not cover everything that is ``true'' about arithmetic
  \item Every attempt to fix it is doomed
  \end{enumerate}
\end{frame}

\begin{frame}
  {Consequences}

  \emph{Some people} have said that G{\"o}del's Incompleteness theorem
  shows that there are fundamental limitations to what computers can
  reason about.

  \pause
  \bigskip

  The reasoning (roughly) goes:
  \begin{enumerate}
  \item Computers can only use effectively generatable axioms
  \item This means that there are truths they cannot prove
  \item Humans can perceive ``real'' truth to see these truths
  \item Therefore, Humans are better than computers, and AI is impossible.
  \end{enumerate}
\end{frame}

\begin{frame}
  {Consequences}

  Two problems with this:

  \begin{enumerate}
  \item Humans only know that the formula generated by G{\"o}del's
    Incompleteness Theorem is true by some \emph{larger}
    axiomatisation we are (maybe implicitly) using. Computers can use
    this axiomatisation.
  \item The theorem depends on the theory being consistent. How do we
    \emph{know} this? Definitely not obvious for Zermelo-Frankel set
    theory.
  \end{enumerate}
\end{frame}

\begin{frame}
  {(In)Completeness?}

  G{\"o}del proved:
  \begin{enumerate}
  \item Completeness ``Everything that is true is provable''
  \item Incompleteness ``There exist true things that are not provable''
  \end{enumerate}

  \bigskip

  Surely a contradiction?
\end{frame}

\begin{frame}
  {(In)Completeness?}

  There is no contradiction.

  \pause
  \bigskip

  Completeness Theorem says that if something is true in \emph{every}
  model of the axioms, it is provable.

  \pause
  \bigskip

  Incompleteness Theorem only gives something that is true for
  \emph{``the''} natural numbers. It might be false in other models.
\end{frame}

\begin{frame}
  {Automating Mathematics?}

  If we can't completely automate arithmetic, then what can we do?

  \begin{enumerate}
  \item Do proof search with a timeout
  \item Restrict to weaker systems to gain decidability, e.g.:
    \begin{itemize}
    \item Pure equality
    \item Linear Arithmetic: only addition, no multiplication
    \end{itemize}
  \end{enumerate}

  \bigskip

  Automated proof for fragments of logic is a large and ongoing topic
  of research, with applications in software engineering, computer
  security, optimisation, ...
\end{frame}

\begin{frame}
  {Summary}

  \begin{enumerate}
  \item Our proof system is sound
  \item If we add excluded middle, it is complete
  \item G{\"o}del's Incompleteness theorem:
    \begin{itemize}
    \item If some axioms can prove basic facts about arithmetic, then
      there are statements that it can neither prove nor disprove.
    \end{itemize}
  \item Not every theory is decidable, but some useful ones are.
  \end{enumerate}
\end{frame}

% \weeksection{Formulas from Programs}

% \newcommand{\pred}[1]{\mathrm{#1}}
% \newcommand{\lit}[1]{\textit{#1}}

% \begin{frame}
%   \sechead{Pre- and Post-conditions}

%   The goal is to establish statements of the form:
%   \begin{displaymath}
%     \{P\}~\texttt{C}~\{Q\}
%   \end{displaymath}
%   where
%   \begin{itemize}
%   \item $P$ is a \emph{precondition}
%   \item \texttt{C} is a program (``command'')
%   \item $Q$ is a \emph{postcondition}
%   \end{itemize}

%   \medskip

%   meaning: ``If the initial state satisfies $P$, then after running
%   \texttt{C}, the final state will satisfy $Q$''.

%   \bigskip

%   \rhighlight{Note}: this says nothing about what happens if
%   \texttt{C} doesn't finish.
% \end{frame}

% \begin{frame}
%   \sechead{Generating Weakest Preconditions}

%   We take programs:
%   \begin{displaymath}
%     \texttt{C}
%   \end{displaymath}
%   and \emph{postconditions} (Predicate logic formulas):
%   \begin{displaymath}
%     Q
%   \end{displaymath}
%   and to compute their \emph{weakest precondition}:
%   \begin{displaymath}
%     \pred{wp}(\texttt{C}, Q)
%   \end{displaymath}
%   and we try to prove that $P \to \pred{wp}(\texttt{C}, Q)$.
% \end{frame}

% \begin{frame}
%   \sechead{Computing formulas for loop-less programs}

%   Given a program \texttt{C} and a postcondition $Q$, we will \\
%   compute the \emph{weakest precondition}, a formula:
%   \begin{displaymath}
%     \mathrm{wp}(\texttt{C}, Q)
%   \end{displaymath}
%   by looking at the structure of \texttt{C}.

%   \bigskip

%   \emph{Weakest precondition} means:
%   \begin{enumerate}
%   \item It is a good precondition: If we can prove
%     $\mathrm{wp}(\texttt{C}, Q)$ about the initial state, then after
%     \texttt{C} terminates, $Q$ will be true.
%   \item It is the \emph{weakest}: If there is another formula $P'$
%     that is also a precondition, then $P' \to \mathrm{wp}(\texttt{C}, Q)$.
%     For example the precondition $x > 0$ is weaker than $x > 1$, because
%     \begin{displaymath}
%       x > 1 \to x > 0
%     \end{displaymath}
%   \end{enumerate}
% \end{frame}

% \begin{frame}
%   \sechead{Weakest Precondition for \texttt{skip}}

%   \begin{displaymath}
%     \mathrm{wp}(\texttt{skip}, Q) = Q
%   \end{displaymath}

%   Why?
%   \begin{enumerate}
%   \item \texttt{skip} does nothing
%   \item So, if we want $Q$ to be true after running \texttt{skip},
%     then it better be true beforehand.
%   \end{enumerate}
% \end{frame}

% \begin{frame}
%   \sechead{Weakest Precondition for \texttt{C$_1$; C$_2$}}
%   \begin{displaymath}
%     \mathrm{wp}(\texttt{C$_1$; C$_2$}, Q) = \mathrm{wp}(\texttt{C$_1$}, \mathrm{wp}(\texttt{C$_2$}, Q))
%   \end{displaymath}

%   Why?
%   \begin{enumerate}
%   \item \texttt{C$_1$; C$_2$} does \texttt{C$_1$} and then \texttt{C$_2$}
%   \item If we want $Q$ to be true at the end, then:
%   \item We want $\mathrm{wp}(\texttt{C$_2$}, Q)$ to be true before running \texttt{C$_2$}, and
%   \item $\mathrm{wp}(\texttt{C$_1$}, \mathrm{wp}(\texttt{C$_2$}, Q))$ before running \texttt{C$_1$}
%   \end{enumerate}
% \end{frame}

% \begin{frame}
%   \sechead{Weakest Precondition for \texttt{x := $t$}}

%   \begin{displaymath}
%     \mathrm{wp}(\texttt{x := $t$}, Q) = Q[x := t]
%   \end{displaymath}

%   Why?
%   \begin{enumerate}
%   \item We want $Q$ to be true after setting $x$ to $t$.
%   \item So we take $Q$ and set $x$ to be $t$ (by substitution).
%   \end{enumerate}

%   \sidenote{This rule was not at all obvious: this, and the whole
%     weakest precondition set up translate programs (which evolve over
%     time) into formulas (which don't).}
% \end{frame}

% \begin{frame}
%   \sechead{Weakest precondition for \texttt{if}}

%   \begin{displaymath}
%     \begin{array}{l}
%       \mathrm{wp}(\texttt{if ($P$) \{ C$_1$ \} else \{ C$_2$ \}}, Q) = \\
%       \qquad
%       \begin{array}[t]{cl}
%         &(P \to \mathrm{wp}(\texttt{C$_1$}, Q)) \\
%         \land&(\lnot P \to \mathrm{wp}(\texttt{C$_2$}, Q))
%       \end{array}
%     \end{array}
%   \end{displaymath}

%   Why?
%   \begin{enumerate}
%   \item If $P$ is true, then we do \texttt{C$_1$}, if $P$ is false, we do \texttt{C$_2$}
%   \item So we need to account for two possibilities:
%     \begin{enumerate}
%     \item $P$ is true, so the precondition will be the one for \texttt{C$_1$}
%     \item $P$ is false, so the precondition will be the one for \texttt{C$_2$}
%     \end{enumerate}
%   \item We \emph{do not} $\lor$ them together. \\
%     \sidenote{The program may do this or that, so we have to prove
%       that \emph{and} that, to cover all the possible cases}
%   \end{enumerate}
% \end{frame}

% \begin{frame}
%   \sechead{What about loops?}

%   \begin{displaymath}
%     \texttt{while ($P$) \{ C \}}
%   \end{displaymath}
%   ``If $P$ is false, stop; otherwise do \texttt{C} and then do the loop again''

%   \pause
%   \bigskip

%   To calculate:
%   \begin{displaymath}
%     \pred{wp}(\texttt{while ($P$) \{C\}}, Q)
%   \end{displaymath}
%   Let's try:
%   \begin{displaymath}
%     \begin{array}{cll}
%        &\pred{wp}(\texttt{while ($P$) \{C\}}, Q) \\
%       =&(\lnot P \to Q) & \textit{if $P$, then loop does not execute} \\
%        &\land (P \to \pred{wp}(\texttt{C; while($P$)\{C\}}, Q)) & \textit{if not $P$, do \texttt{C} and try loop again}\\
%       =&(\lnot P \to Q)\\
%        &\multicolumn{2}{l}{\land (P \to \pred{wp}(\texttt{C}, \pred{wp}(\texttt{while ($P$) \{C\}}, Q)))}
%     \end{array}
%   \end{displaymath}
%   But now if we expand
%   \begin{displaymath}
%     \pred{wp}(\texttt{while ($P$) \{C\}}, Q)
%   \end{displaymath}
%   again, we are stuck in a loop...
% \end{frame}

% \begin{frame}
%   \sechead{Cutting the Gordian Knot}

%   If we want to work out some formula $X$ for
%   \begin{displaymath}
%     \pred{wp}(\texttt{while ($P$) \{C\}}, Q)
%   \end{displaymath}
%   We know, by the reasoning on the previous slide, that it must imply:
%   \begin{displaymath}
%     (\lnot P \to Q)\land (P \to \pred{wp}(\texttt{C}, \pred{wp}(\texttt{while ($P$) \{C\}}, Q)))
%   \end{displaymath}
%   so we just ask the programmer to give an $X$ such that:
%   \begin{displaymath}
%     X \to ((\lnot P \to Q)\land (P \to \pred{wp}(\texttt{C}, X)))
%   \end{displaymath}
%   and set $\pred{wp}(\texttt{while ($P$) \{C\}}, Q) = X$.

%   \bigskip

%   \begin{itemize}
%   \item The $X$ is specific to this loop and $Q$.
%   \item $X$ is the \emph{loop invariant}
%   \end{itemize}
% \end{frame}

% \begin{frame}
%   \sechead{Annotated Loops}

%   To allow weakest precondition calculation, we annotate loops with
%   their invariants:

%   \begin{displaymath}
%     \texttt{while$^{I}$ ($P$) \{ C \}}
%   \end{displaymath}

%   Now we have:
%   \begin{displaymath}
%     \begin{array}{l}
%       \pred{wp}(\texttt{while$^{I}$ ($P$) \{ C \}}, Q) = I
%     \end{array}
%   \end{displaymath}
%   and we also have to prove:
%   \begin{displaymath}
%     \begin{array}{cl}
%       1.&(\lnot P \land I) \to Q\\
%       2.&(P \land I) \to \pred{wp}(\texttt{C},I)
%     \end{array}
%   \end{displaymath}
% \end{frame}


% \weeksection{Logic in Computer Science}

% \begin{frame}
%   What have we done?

%   \begin{itemize}
%   \item Week 1 : Propositional Logic
%   \item Weeks 2\&3 : Logical Modelling
%   \item Weeks 4\&5 : Deductive Systems
%   \item Weeks 6\&7: Predicate Logic, syntax and proofs
%   \item Week 8 : Predicate Logic, semantics
%   \item Week 9 : Arithmetic and Induction
%   \end{itemize}
% \end{frame}

% \begin{frame}
%   \frametitle{Logical Modelling}

%   \begin{displaymath}
%     (a \lor b \lor \lnot c) \land \cdots
%   \end{displaymath}

%   \begin{itemize}
%   \item Solutions as True/False assignments
%   \item Constraints as logical formulas
%   \item Finding solutions as \emph{search}
%   \end{itemize}

%   Going further:
%   \begin{itemize}
%   \item Computation as search (Prolog, ...)
%   \item Satisfiability Modulo Theory (SMT) solvers
%   \item Specialised solvers
%   \end{itemize}
% \end{frame}

% \begin{frame}
%   \frametitle{Deductive systems}

%   \begin{displaymath}
%     \inferrule*
%     {\mathit{premise}_1 \cdots \mathit{premise}_n}
%     {\mathit{conclusion}}
%   \end{displaymath}

%   \begin{itemize}
%   \item Rules as steps of inference
%   \item Proofs as rules arranged into a tree
%   \end{itemize}

%   Going further:
%   \begin{itemize}
%   \item Defining Programming Languages:
%     \begin{itemize}
%     \item What programs are well-typed?
%     \item What do programs do?
%     \end{itemize}
%   \item Programs as proof search (Prolog, Datalog)
%   \end{itemize}
% \end{frame}

% \begin{frame}
%   \frametitle{Predicate Logic: Syntax and Proofs}

%   \begin{displaymath}
%     \forall x. \exists y. \cdots
%   \end{displaymath}

%   \begin{itemize}
%   \item Statements about \emph{all} or \emph{some}
%   \item Quantifiers as a ``game''
%   \end{itemize}

%   Going further:
%   \begin{itemize}
%   \item Software specification
%   \item
%   \end{itemize}
% \end{frame}

% \begin{frame}
%   \frametitle{Predicate Logic: Semantics}

%   \begin{displaymath}
%     \mathit{within} = \{ (\mathsf{glasgow}, \mathsf{scotland}), \cdots \}
%   \end{displaymath}

%   \begin{itemize}
%   \item Models of Predicate Logic $\approx$ databases
%   \end{itemize}

%   Going further:
%   \begin{itemize}
%   \item Deductive databases
%   \item Probabilistic databases
%   \item Temporal databases
%   \item Model generation
%   \end{itemize}
% \end{frame}

% \begin{frame}
%   \frametitle{
% \end{frame}

% \begin{frame}
%   \begin{center}
%     What we didn't cover
%   \end{center}
% \end{frame}

% \begin{frame}
%   \frametitle{Modal Logic}

%   \sechead{Propositional Logic}
%   Make statements about ``now'', ``timeless'', ``one possibility''

%   \begin{displaymath}
%     \mathit{Raining} \lor \mathit{Sunny}
%   \end{displaymath}

%   \pause
%   \bigskip

%   \sechead{Modal Logic}
%   Statements like:
%   \begin{enumerate}
%   \item It is \textbf{necessary} that $P$
%   \item It is \textbf{possible} that $P$
%   \item $P$ will be true \textbf{tomorrow}
%   \item $P$ will be true \textbf{forever}
%   \item $P$ will be true \textbf{eventually}
%   \end{enumerate}
%   \medskip

%   Reasoning about \emph{possible worlds}, \emph{possible futures}.
% \end{frame}

% \begin{frame}
%   \frametitle{Program Verification}


% \end{frame}

\end{document}
