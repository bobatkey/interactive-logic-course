% -*- TeX-engine: xetex -*-

\documentclass[xetex,aspectratio=169,14pt,hyperref={pdfpagelabels=true,pdflang={en-GB}}]{beamer}

\usepackage[sci,noslidestrathidentity]{strathclyde}
\strathsetidentity{Department of}{Computer \& Information Sciences}

\usepackage{lmodern}
\usepackage{subscript}
\usepackage{url}
\usepackage{pifont}
\usepackage{csquotes}
\usepackage{mathpartir}
\usepackage{stmaryrd}
\usepackage{multirow}
\usepackage{euler}
\usepackage[normalem]{ulem}

% 2013-09-08 SU adding xetex specifics
% http://www.woggie.net/2008/07/16/beamer-pdftex-and-xetex/
\usepackage{xltxtra}

% 2013-09-08 SU http://robjhyndman.com/hyndsight/xelatex/
\defaultfontfeatures{Ligatures=TeX}

% 2015-12-09 SU table beautified
\renewcommand{\arraystretch}{1.2}
\usepackage{booktabs}

%MGK compatibility
\newcommand{\hh}[1]
  {\medskip\textbf{\large #1}}
\newcommand{\pdu}[3]
  {#1\rightarrow #2 : &\ & \makebox[70mm][l]{$#3$}}



\setlength{\marginparwidth}{2cm}
\usepackage{todonotes}%[disable]
\let\OldTodo\todo
\renewcommand{\todo}{\OldTodo[inline]}%
\newcommand{\todolater}[1]{}% Things to do for next year


\DeclareTextCommandDefault{\nobreakspace}{\leavevmode\nobreak\ }
\usepackage{rotating}

\usepackage{pdfcomment}% To add alt text for images using \pdftooltip{}

\usepackage{appendixnumberbeamer}

\newcommand{\messageframe}[1]{\begin{frame}\begin{center}\Huge #1\end{center}\end{frame}}
\newcommand{\sechead}[1]{{\bf #1} \\}
\newcommand{\examplehead}[1]{{\bf Example:} {\it\textcolor{red!90}{#1}} \\}
\newcommand{\eqnote}[1]{\hspace{3cm}\textit{#1}}
\newcommand{\sidenote}[1]{\qquad {\footnotesize \textcolor{black!60}{(#1)}}}
\newcommand{\sem}[1]{\llbracket #1 \rrbracket}
\newcommand{\true}{\mathsf{T}}
\newcommand{\false}{\mathsf{F}}

\def\strikeafter<#1>#2{\temporal<#1>{#2}{\sout{#2}}{\sout{#2}}}


%\newcommand{\rhighlight}{\textcolor{titlered}}
\newcommand{\rhighlight}{\textbf}
\newcommand{\highlight}{\textbf}

% \setmainfont{Linux Biolinum O}
\setmainfont{LinBiolinum}[
Path=,
UprightFont = *_R.otf ,
BoldFont = *_RB.otf ,
ItalicFont = *_RI.otf
]

\setbeamertemplate{navigation symbols}{}
%\usecolortheme[rgb={0.8,0,0}]{structure}
\usefonttheme{serif}
\usefonttheme{structurebold}
\setbeamercolor{description item}{fg=black}


\author[Atkey]{Dr.~Robert Atkey}
\institute[Strathclyde]{Computer \& Information Sciences}
\date[]{}

\newcommand{\weeksection}[1]{%
  \section{\thetitle{}, Part~\thesection : #1}
  \begin{frame}
    \begin{center}
      \textcolor{black!60}{\thetitle{}, Part \thesection}\\
      {\Huge #1}
    \end{center}
  \end{frame}}

\newcommand{\weektitle}[2]{\def\thetitle{#2}
\title[CS208 - Week #1]{CS208 (Semester 1) Week #1 : #2}}

\newcommand{\assigned}{:}

\newcommand{\forcedto}{\assigned_f}
\newcommand{\decideto}{\assigned_d}


\weektitle{7}{Predicate Logic : Natural Deduction}

\begin{document}

\frame{\titlepage}

\weeksection{Upgrading Natural Deduction}

\begin{frame}[t]
  {Tracking free variables}

  \bigskip

  We are going to prove things like:
  \begin{displaymath}
    \vdash \forall x. (p(x) \land q(x)) \to p(x)
  \end{displaymath}

  \bigskip

  This will mean we will have proof states like:
  \begin{displaymath}
    \dots \vdash (p(x) \land q(x)) \to p(x)
  \end{displaymath}

  We need to keep track of variables as well as assumed formulas to
  the left of the $\vdash$ ``turnstile''.
\end{frame}

\begin{frame}
  {Judgements}

  Proving:
  \begin{displaymath}
    \underbrace{P_1, x_1, \dots, x_i, P_j, \dots, x_m, P_n}_{\textit{assumptions and variables}} \vdash \underbrace{Q}_{\textit{conclusion}}
  \end{displaymath}

  Focused:
  \begin{displaymath}
    \underbrace{P_1, x_1, \dots, x_i, P_j, \dots, x_m, P_n}_{\textit{assumptions and variables}}~[\underbrace{P}_{\textit{focus}}] \vdash \underbrace{Q}_{\textit{conclusion}}
  \end{displaymath}

  Note:
  \begin{enumerate}
  \item We never focus on a variable, only formulas
  \item Each $P_j$ only contains free variables that appear to the
    \emph{left} of it
  \end{enumerate}
\end{frame}

\begin{frame}[t]
  \sechead{Well-scoped terms and formulas}

  If we have a list of variables and assumptions (a ``context''):
  \begin{displaymath}
    \Gamma = P_1, x_1, \dots, x_i, P_j, \dots, x_m, P_n
  \end{displaymath}
  \textcolor{black!60}{$\Gamma$ is the name we're giving to the list}

  \bigskip

  \begin{itemize}
  \item A formula $P$ is \emph{well-scoped in $\Gamma$} if all the
    free variables of $P$ appear in $\Gamma$.
  \item A term $t$ is \emph{well-scoped in $\Gamma$} if all the
    variables of $t$ appear in $\Gamma$.
  \end{itemize}

  \bigskip

  \begin{itemize}
  \item All formulas in $\Gamma$ must be well-scoped by the variables
    to their left (same condition as previous slide).
  \item The focus and conclusion must always be well-scoped in
    $\Gamma$.
  \end{itemize}
\end{frame}

\begin{frame}
  {Well-scoped terms and formulas}

  Are the following well-scoped?

  \bigskip

  \begin{enumerate}
  \item Context: $x$\quad Formula: $\forall y. P(y) \to Q(y)$\\
    \pause
    Yes. The variable $y$ is bound in the formula.
    \pause
  \item Context: $x$\quad Formula: $\forall y. P(y) \to Q(x,y)$\\
    \pause Yes. The variable $y$ is bound in the formula, and the free
    variable $x$ is in the context.
  \end{enumerate}
\end{frame}

\begin{frame}
  {Well-scoped terms and formulas}
  Are the following well-scoped?

  \bigskip

  \begin{enumerate}
  \item Context: \emph{empty}\quad Formula: $\forall y. P(y) \to Q(x,y)$\\
    \pause No. The variable $y$ is bound in the formula, but the free
    variable $x$ is not in the context.
    \pause
  \item Context: \emph{empty}\quad Term: $x+1$\\
    \pause No. The variable $x$ is free in the term but is not in the
    context.  \pause
  \end{enumerate}
\end{frame}

\begin{frame}
  {Well-scoped Judgements}
  Is the following well-scoped?

  \bigskip

  \begin{enumerate}
  \item Is this judgement well-scoped:
    \begin{displaymath}
      x, y~[P(x,y)] \vdash Q(x)
    \end{displaymath}
    \pause Yes. The free variables of the focus and conclusion are
    $x, y$, which are in the context.

  \end{enumerate}
\end{frame}

\begin{frame}
  {Well-scoped Judgements}
  Is the following well-scoped?

  \bigskip

  \begin{enumerate}
  \item Is this judgement well-scoped:
    \begin{displaymath}
      x~[P(x,y)] \vdash Q(x)
    \end{displaymath}
    \pause No. The free variables of the focus and conclusion are
    $x, y$, but $y$ is not in the context.

  \end{enumerate}
\end{frame}

\begin{frame}
  {Well-scoped Judgements}
  Is the following well-scoped?

  \bigskip

  \begin{enumerate}
  \item Is this judgement well-scoped:
    \begin{displaymath}
      x, Q(x), y~[P(x,y)] \vdash Q(y)
    \end{displaymath}
    \pause Yes. Each variable appears before (reading left to right)
    it is used.
  \end{enumerate}
\end{frame}

\begin{frame}
  {Well-scoped Judgements}
  Is the following well-scoped?

  \bigskip

  \begin{enumerate}
  \item Is this judgement well-scoped:
    \begin{displaymath}
      \forall x. Q(x), y~[P(x,y)] \vdash Q(y)
    \end{displaymath}
    \pause No. The $x$ in the first $Q(x)$ is OK, but the $x$ in
    $P(x,y)$ has not been declared in scope.
  \end{enumerate}
\end{frame}

\begin{frame}
  {Summary}

  \begin{enumerate}
  \item We started to upgrade Natural Deduction to Predicate Logic
  \item We need to manage the \emph{scope} of variables
  \item To do so, we add them to the context
  \item Variables may only be used by formulas to their right
  \end{enumerate}
\end{frame}

% \begin{frame}
%   \sechead{Bound and Free Variables}
%   The Quantifiers are \emph{binders}.

%   \bigskip

%   In the formula:
%   \begin{displaymath}
%     \exists k. (x = k + k) \lor (x = k + k + 1)
%   \end{displaymath}
%   \begin{enumerate}
%   \item The variable $x$ is \emph{free}
%   \item The variable $k$ is \emph{bound}
%   \end{enumerate}

%   \pause

%   Bound variables can be renamed as we like:
%   \begin{displaymath}
%     \exists j. (x = j + j) \lor (x = j + j + 1)
%   \end{displaymath}

% \end{frame}

\weeksection{Substitution}

\begin{frame}
  {From General to Specific}

  We will have \emph{general} assumptions like:
  \begin{displaymath}
    \forall x. \mathrm{human}(x) \to \mathrm{mortal}(x)
  \end{displaymath}

  \bigskip

  And we want to \emph{specialise} (or \emph{instantiate}) to:
  \begin{displaymath}
    \mathrm{human}(\mathsf{socrates}()) \to \mathrm{mortal}(\mathsf{socrates}())
  \end{displaymath}
\end{frame}

\begin{frame}
  {Substitution}
  The notation
  \begin{displaymath}
    P[x:=t]
  \end{displaymath}
  means ``replace all \emph{free} occurrences of $x$ in $P$ with $t$''.
  \begin{itemize}
  \item $x$ is a \emph{variable}
  \item $P$ is a \emph{formula}
  \item $t$ is a \emph{term}
  \end{itemize}


  \bigskip

  \textcolor{black!60}{But there is a subtlety...}
\end{frame}

\begin{frame}
  {Substitution Examples}

  \begin{displaymath}
    \begin{array}{cl}
      &(\mathrm{mortal}(x))[x := \mathsf{socrates}()]\\
      \Longrightarrow&\mathrm{mortal}(\mathsf{socrates}())\\
    \end{array}
  \end{displaymath}
\end{frame}

\begin{frame}
  {Substitution Examples}

  \begin{displaymath}
    \begin{array}{cl}
      &(\forall y. \mathrm{weatherIs}(d,y) \to \mathrm{weatherIs}(\mathsf{dayAfter}(d),y))[d:=\mathsf{tuesday}]\\
      \Longrightarrow&\forall y. \mathrm{weatherIs}(\mathsf{tuesday},y) \to \mathrm{weatherIs}(\mathsf{dayAfter}(\mathsf{tuesday}),y)\\
    \end{array}
  \end{displaymath}
\end{frame}

\begin{frame}
  {Substitution Examples}

  \begin{displaymath}
    \begin{array}{cl}
      &(\exists y. \mathrm{sameElements}(x,y) \land \mathrm{sorted}(y))[x:=\mathsf{cons}(z_1,\mathsf{cons}(z_2,\mathsf{nil}))]\\
      \Longrightarrow&\exists y. \mathrm{sameElements}(\mathsf{cons}(z_1,\mathsf{cons}(z_2,\mathsf{nil})),y) \land \mathrm{sorted}(y)\\
    \end{array}
  \end{displaymath}
\end{frame}

\begin{frame}
  {Substitution Examples}
  \begin{displaymath}
    \begin{array}{cl}
      &(\forall y. x+y=y+x)[x:=z-z] \\
      \Longrightarrow&\forall y. (z-z)+y = y+(z-z)
    \end{array}
  \end{displaymath}
\end{frame}

\begin{frame}[t]
  {Accidental Name Capture}

  If we substitute naively, then we produce nonsense:
  \begin{enumerate}
  \item $\exists y. \mathrm{sameElements}(x,y)$ \\
    \emph{``there exists a $y$ that has the same elements as $x$''}\\
    ~\\
  \item $(\exists y. \mathrm{sameElements}(x,y))[x:=\mathsf{append}(y,[1,2])]$\\
    \emph{``replace $x$ by the list $\mathsf{append}(y,[1,2])$''}\\
    ~\\
  \item $\exists y. \mathrm{sameElements}(\mathsf{append}(y,[1,2]),y)$\\
    \emph{``there exists a $y$ that has the same elements as $y + [1,2]$?''}
  \end{enumerate}
\end{frame}

\begin{frame}
  {Capture Avoidance}

  \bigskip

  \textbf{Solution: Rename bound variables}
  \begin{displaymath}
    \begin{array}{cl}
      &(\exists y. \mathrm{sameElements}(x,y))[x:=\mathsf{append}(y,[1,2])]\\
      \Longrightarrow&(\exists z. \mathrm{sameElements}(x,z))[x:=\mathsf{append}(y,[1,2])]\\
      \Longrightarrow&\exists z. \mathrm{sameElements}(\mathsf{append}(y,[1,2]),z)
    \end{array}
  \end{displaymath}
\end{frame}

\begin{frame}
  {Capture Avoiding Substitution}

  When working out
  \begin{displaymath}
    P[x := t]
  \end{displaymath}
  If any of the variables in $t$ are bound in $P$ then rename them
  before doing the substitution.
\end{frame}

\begin{frame}
  {Substitution Examples}

  \begin{enumerate}
  \item $P(x,y)[x:=y+y]$\pause \quad $= P(y+y,y)$
    \pause
    \bigskip
  \item $P(x,y)[y:=y+y]$\pause \quad $= P(x,y+y)$
    \pause
    \bigskip
  \item $(\forall x. P(x,y))[x:=y+y]$\pause \quad $= \forall x. P(x,y)$
  \end{enumerate}
\end{frame}


\begin{frame}
  {Substitution Examples}

  \begin{enumerate}
  \item $(\forall x. P(x,y))[y:=x+x]$\pause \quad $= \forall z. P(z,x+x)$ \\
    \qquad \textcolor{black!60}{Renaming!}
    \pause
    \bigskip
  \item $(\forall x. P(x,y) \to (\exists z. Q(y,z)))[y:=z+z]$\pause\\
    $= \forall x. P(x,z+z) \to (\exists w. Q(z+z,w))$\\
    \qquad \textcolor{black!60}{Renaming!}
    \pause
    \bigskip
  \item $(\forall x. P(x,z) \to (\exists z. Q(y,z)))[z:=x+x]$\pause\\
    $= \forall w. P(w,x+x) \to (\exists z. Q(y,z))$\\
    \qquad \textcolor{black!60}{Renaming! and no substitution of the final $z$}
  \end{enumerate}
\end{frame}

\begin{frame}
  {Summary}

  \begin{itemize}
  \item Substitution
    \begin{displaymath}
      P [x := t]
    \end{displaymath}
    is how we go from the general $x$ to the specific $t$.
  \item We need to be careful to rename bound variables to avoid
    accidental name capture.
  \end{itemize}
\end{frame}

\weeksection{Rules for ``Forall''}

\begin{frame}[t]
  \sechead{What does $\forall x. P$ mean?}
  \sidenote{assuming a domain of discourse}

  \bigskip

  \emph{Answer 1} : it means for all individuals ``$\mathsf{a}$'', $P[x:=\mathsf{a}]$ is true.\\
  \sidenote{we think of ``for all'' as an infinite conjunction}

  \pause
  \bigskip

  \emph{Answer 2} : thinking about proofs:

  \medskip

  To \emph{prove} a $\forall x. P$:
  \begin{itemize}
  \item We must prove $P[x:=x_0]$ for a \emph{general} $x_0$.
  \item The $x_0$ stands in for any ``$\mathsf{a}$'' that might be chosen.
  \end{itemize}

  \bigskip

  To \emph{use} a proof of $\forall x. P$:
  \begin{itemize}
  \item We can \emph{choose} any $t$ we like for $x$, and get $P[x:=t]$
  \end{itemize}
\end{frame}

\begin{frame}
  \sechead{Introduction rule}

  \begin{displaymath}
    \inferrule* [right=Introduce$\forall$]
    {\Gamma, x_0 \vdash Q[x:=x_0]}
    {\Gamma \vdash \forall x. Q}
  \end{displaymath}

  \bigskip
  \pause

  ``To prove $\forall x. Q$, we prove $Q[x:=x_0]$, assuming an arbitrary $x_0$.''
\end{frame}

\begin{frame}

  \begin{displaymath}
    \inferrule* [right=Introduce]
    {\inferrule* [Right=Introduce]
      {\inferrule* [Right=Use]
        {\inferrule* [Right=First]
          {\inferrule* [Right=Done]
            { }
            {\mathit{x}, \mathrm{P}(\mathit{x}) \land \mathrm{Q}(\mathit{x})~[\mathrm{P}(\mathit{x})] \vdash \mathrm{P}(\mathit{x})}
          }
          {\mathit{x}, \mathrm{P}(\mathit{x}) \land \mathrm{Q}(\mathit{x})~[\mathrm{P}(\mathit{x}) \land \mathrm{Q}(\mathit{x})] \vdash \mathrm{P}(\mathit{x})}
        }
        {\mathit{x}, \mathrm{P}(\mathit{x}) \land \mathrm{Q}(\mathit{x}) \vdash \mathrm{P}(\mathit{x})}
      }
      {\mathit{x} \vdash (\mathrm{P}(\mathit{x}) \land \mathrm{Q}(\mathit{x})) \to \mathrm{P}(\mathit{x})}
    }
    { \vdash \forall \mathit{x}. (\mathrm{P}(\mathit{x}) \land \mathrm{Q}(\mathit{x})) \to \mathrm{P}(\mathit{x})}
  \end{displaymath}
\end{frame}

\begin{frame}
  \sechead{Elimination}

  \begin{displaymath}
    \inferrule* [right=Instantiate]
    {\Gamma~[P[x:=t]] \vdash Q}
    {\Gamma~[\forall x. P] \vdash Q}
  \end{displaymath}

  \bigskip

  (side condition: $t$ is well-scoped in $\Gamma$)

  \bigskip
  \pause

  ``If we have $P$ for all $x$, then we can pick any well-scoped $t$
  we like to stand in for it.''

\end{frame}

\begin{frame}

  \begin{displaymath}
    \inferrule* [right=Introduce]
    {\inferrule* [Right=Introduce]
      {\inferrule* [Right=Use]
        {\inferrule* [Right=Instantiate]
          {\inferrule* [Right=Apply]
            {\inferrule* [right=Use]
              {\inferrule* [Right=Done]
                { }
                {\Gamma~[\mathrm{h}(\mathsf{s}())] \vdash \mathrm{h}(\mathsf{s}())}
              }
              {\Gamma \vdash \mathrm{h}(\mathsf{s}())}
              \\
              \inferrule* [Right=Done]
              { }
              {\Gamma~[\mathrm{m}(\mathsf{s}())] \vdash \mathrm{m}(\mathsf{s}())}
            }
            {\Gamma~[\mathrm{h}(\mathsf{s}()) \to \mathrm{m}(\mathsf{s}())] \vdash \mathrm{m}(\mathsf{s}())}
          }
          {\Gamma~[\forall \mathit{x}. \mathrm{h}(\mathit{x}) \to \mathrm{m}(\mathit{x})] \vdash \mathrm{m}(\mathsf{s}())}
        }
        {\Gamma \vdash \mathrm{m}(\mathsf{s}())}
      }
      {\forall \mathit{x}. \mathrm{h}(\mathit{x}) \to \mathrm{m}(\mathit{x}) \vdash \mathrm{h}(\mathsf{s}()) \to \mathrm{m}(\mathsf{s}())}
    }
    { \vdash (\forall \mathit{x}. \mathrm{h}(\mathit{x}) \to \mathrm{m}(\mathit{x})) \to \mathrm{h}(\mathsf{s}()) \to \mathrm{m}(\mathsf{s}())}
  \end{displaymath}
  where
  $\Gamma = \forall \mathit{x}. \mathrm{h}(\mathit{x}) \to \mathrm{m}(\mathit{x}), \mathrm{h}(\mathsf{s}())$
\end{frame}

\begin{frame}
  {Summary}

  \begin{itemize}
  \item To prove $\forall x. P(x)$, we must prove $P(x_0)$ for a general $x_o$.
  \item To use $\forall x. P(X)$, we get to choose the $t$ we use for $x$.
  \end{itemize}
\end{frame}

\weeksection{Rules for ``Exists''}

\begin{frame}[t]
  \sechead{What does $\exists x. P$ mean?}
  \sidenote{assuming a domain of discourse}

  \bigskip

  \emph{Answer 1 } : there is at least one ``$\mathsf{a}$'' such that $P[x:=\mathsf{a}]$ is true.\\
  \sidenote{we think of ``exists'' as an infinite disjunction}

  \pause
  \bigskip

  \emph{Answer 2} : thinking about proofs:

  \medskip

  To \emph{prove} a $\exists x.P$:
  \begin{itemize}
  \item We must provide a \emph{witness} term $t$ such that $P[x:=t]$.
  \end{itemize}

  \bigskip

  To \emph{use} a proof of $\exists x. P$:
  \begin{itemize}
  \item We have to work with an arbitrary $x_0$ and all we know is $P[x:=x_0]$.
  \end{itemize}
\end{frame}

\begin{frame}
  \sechead{Introduction}

  \begin{displaymath}
    \inferrule* [right=Exists]
    {\Gamma \vdash P[x:=t]}
    {\Gamma \vdash \exists x. P}
  \end{displaymath}

  (side condition: $t$ is well-scoped in $\Gamma$)

  \bigskip

  ``To prove $\exists x. P$, we have to provide a witness $t$ for $x$,
  and show that $P[x:=t]$''
\end{frame}

\begin{frame}

  {\small
  \begin{displaymath}
    \inferrule* [right=Introduce]
    {\inferrule* [Right=Exists]
      {\inferrule* [Right=Use]
        {\inferrule* [Right=Done]
          { }
          {\mathrm{human}(\mathsf{socrates}())~[\mathrm{human}(\mathsf{socrates}())] \vdash \mathrm{human}(\mathsf{socrates}())}
        }
        {\mathrm{human}(\mathsf{socrates}()) \vdash \mathrm{human}(\mathsf{socrates}())}
      }
      {\mathrm{human}(\mathsf{socrates}()) \vdash \exists \mathit{x}. \mathrm{human}(\mathit{x})}
    }
    { \vdash \mathrm{human}(\mathsf{socrates}()) \to (\exists \mathit{x}. \mathrm{human}(\mathit{x}))}
  \end{displaymath}}

\end{frame}

\begin{frame}
  \sechead{Elimination}

  \begin{displaymath}
    \inferrule* [right=Unpack]
    {\Gamma, x_0, P[x:=x_0] \vdash Q}
    {\Gamma~[\exists x. P] \vdash Q}
  \end{displaymath}

  ``To use $\exists x. P$, we get some arbitrary $x_0$ that we know
  $P[x:=x_0]$ about.''
\end{frame}

\begin{frame}

  {\small
    \begin{displaymath}
      \inferrule* [right=Introduce]
      {\inferrule* [Right=Use]
        {\inferrule* [Right=Unpack]
          {\inferrule* [Right=Exists]
            {\inferrule* [Right=Use]
              {\inferrule* [Right=First]
                {\inferrule* [Right=Done]
                  { }
                  {\exists \mathit{x}. \mathrm{h}(\mathit{x}) \land \mathrm{m}(\mathit{x}), \mathit{ali}, \mathrm{h}(\mathit{ali}) \land \mathrm{m}(\mathit{ali})~[\mathrm{h}(\mathit{ali})] \vdash \mathrm{h}(\mathit{ali})}
                }
                {\exists \mathit{x}. \mathrm{h}(\mathit{x}) \land \mathrm{m}(\mathit{x}), \mathit{ali}, \mathrm{h}(\mathit{ali}) \land \mathrm{m}(\mathit{ali})~[\mathrm{h}(\mathit{ali}) \land \mathrm{m}(\mathit{ali})] \vdash \mathrm{h}(\mathit{ali})}
              }
              {\exists \mathit{x}. \mathrm{h}(\mathit{x}) \land \mathrm{m}(\mathit{x}), \mathit{ali}, \mathrm{h}(\mathit{ali}) \land \mathrm{m}(\mathit{ali}) \vdash \mathrm{h}(\mathit{ali})}
            }
            {\exists \mathit{x}. \mathrm{h}(\mathit{x}) \land \mathrm{m}(\mathit{x}), \mathit{ali}, \mathrm{h}(\mathit{ali}) \land \mathrm{m}(\mathit{ali}) \vdash \exists \mathit{x}. \mathrm{h}(\mathit{x})}
          }
          {\exists \mathit{x}. \mathrm{h}(\mathit{x}) \land \mathrm{m}(\mathit{x})~[\exists \mathit{x}. \mathrm{h}(\mathit{x}) \land \mathrm{m}(\mathit{x})] \vdash \exists \mathit{x}. \mathrm{h}(\mathit{x})}
        }
        {\exists \mathit{x}. \mathrm{h}(\mathit{x}) \land \mathrm{m}(\mathit{x}) \vdash \exists \mathit{x}. \mathrm{h}(\mathit{x})}
      }
      { \vdash (\exists \mathit{x}. \mathrm{h}(\mathit{x}) \land \mathrm{m}(\mathit{x})) \to (\exists \mathit{x}. \mathrm{h}(\mathit{x}))}
    \end{displaymath}}
\end{frame}

%\titlecard{Comparing \\the Quantifiers \\to\\ the Connectives}

\begin{frame}
  {Comparing $\land$ and $\forall$}

  \rhighlight{Introduction}
  \begin{mathpar}
    \inferrule* [right=Split]
    {\Gamma \vdash P_1 \\ \Gamma \vdash P_2}
    {\Gamma \vdash P_1 \land P_2}

    \inferrule* [right=$\forall$-I]
    {\Gamma, x_0 \vdash P[x:=x_0]}
    {\Gamma \vdash \forall x. P}
  \end{mathpar}
  For $\land$, we have to prove $P_i$, no matter what $i$ is. For
  $\forall$, we have to prove $P[x:=x_0]$, no matter what $x_0$ is.
\end{frame}

\begin{frame}
  {Comparing $\land$ and $\forall$}

  \rhighlight{Elimination}
  \begin{mathpar}
    \inferrule* [right=First]
    {\Gamma~[P_1] \vdash Q}
    {\Gamma~[P_1 \land P_2] \vdash Q}

    \inferrule* [right=Second]
    {\Gamma~[P_2] \vdash Q}
    {\Gamma~[P_1 \land P_2] \vdash Q}

    \inferrule* [right=Instantiate]
    {\Gamma~[P[x:=t]] \vdash Q}
    {\Gamma~[\forall x. P] \vdash Q}
  \end{mathpar}
  For $\land$, we choose $1$ or $2$. For $\forall$, we choose $t$.
\end{frame}

\begin{frame}
  {Comparing $\lor$ and $\exists$}

  \rhighlight{Introduction}
  \begin{mathpar}
    \inferrule* [right=Left]
    {\Gamma \vdash P_1}
    {\Gamma \vdash P_1 \lor P_2}

    \inferrule* [right=Right]
    {\Gamma \vdash P_2}
    {\Gamma \vdash P_1 \lor P_2}

    \inferrule* [right=Exists]
    {\Gamma \vdash P[x:=t]}
    {\Gamma \vdash \exists x. P}
  \end{mathpar}
  For $\lor$, we choose which of $1$ or $2$ we want. For $\exists$, we
  choose the witnessing term $t$.
\end{frame}

\begin{frame}
  {Comparing $\lor$ and $\exists$}

  \rhighlight{Elimination}
  \begin{mathpar}
    \inferrule* [right=Cases]
    {\Gamma, P_1 \vdash Q \\
      \Gamma, P_2 \vdash Q}
    {\Gamma~[P_1 \lor P_2] \vdash Q}

    \inferrule* [right=Unpack]
    {\Gamma, x_0, P[x:=x_0] \vdash Q}
    {\Gamma~[\exists x. P] \vdash Q}
  \end{mathpar}
  For $\lor$, we must deal with $1$ or $2$. For $\exists$, we must
  cope with any $x_0$.
\end{frame}

\begin{frame}
  {Summary}

  \begin{itemize}
  \item To prove $\exists x. P(x)$ we must give a witness $t$ and prove $P(t)$.
  \item To use $\exists x. P(X)$ we get to assume there is some $y$ and $P(y)$.
  \end{itemize}
\end{frame}

\weeksection{Using the interactive prover}


\end{document}
